\chapter{Introduction}\label{ch:introduction}

This report details a project done in Wireless Sensor Networks (WSN) with a mini-project called "To hop or to hop". The project follows the original idea of determining when to relay packets in a wireless network, but with a little twist to it - instead of placing the receiving node in a fixed position, we place it on an athlete running a marathon on a oval-formed running track in order to measure his heart rate every minute. This information is going to be transfered to a sink (base station) over a wireless connection with an added two relay stations in between that help ensure sufficient network coverage of the track. To design the track for our purpose, we will employ knowledge of fading, radio wave propagation and received input power (dbM) in a wireless node.

\noindent We will look to create an effective protocol that can determine when to communicate directly with the node carried on the athlete/runner and when it is best to use one of the relays. This protocol will take signal strength and energy consumption into consideration. For inter-node connectivity, we will design and implement the data-link layer stop-and-wait ARQ protocol on all nodes.\footnote{\cite{Ieee}}

\noindent With reference to the mini-project presentation, we use telosb nodes all running TinyOS with a packet size of 128 bytes. The RF transceiver is a single-chip 2.4 GHz IEEE 802.15.4 compliant CC2420 with a data rate of 250 kbps.

\subsection{The protocol: To hop or not to hop}

The base station will be requesting the heart rate from the runner node with a packet size of 128 bytes and save it to memory. When the runner node is out of range, the north or south relay station will receive the request from the base station and relay it to the runner. We will look at energy costs and signal strength when deciding to hop or not.

\section{Energy consumption}\label{sc:powerConsumption}
\subsection{Runner node}
%TODO: Review this. What does it mean "Different levels of energy consumption will be determined by the chosen protocol"? WIll protocol work based on whats best energy-wise or tell us how much energy we use?
The base station will be requesting data from the runner at a periodic transmission rate. Different levels of energy consumption will be determined by the chosen protocol, but in any case, the power consumption will be considered over a period of one second, constant. The less energy consumption of the runner will give longer individual lifetime and runtime for it, but low signal strength of node A might not give a lowest possible system power consumption. Depending on the needed quality of the received package, e.g. -84dBm, a cut of distance will be calculated and measured. Distance measuring will be limited by interference providing a need for a scalable transfer function estimate and an average over multiple measurements. Life time of node A will be considered when half of the battery capacity is used.

\subsection{Base station, north and south relay station}\label{sc:relayStations}

Idle time, receiving and transmitting power consumption will be calculated and measure. When out of range the “pole” stations will in theory go to an idle state to save power. When in transmitting mode different measurements will be conducted depending on the chosen protocol. E.g. firm or no handshakes between pole station and base station will be measured leading to different possible distances between jumps.

\subsection{Sink: Base Station}\label{sc:sinkBaseStation}
The required detail of information needed to give a good user estimate will raise the question of acceptable package loss. Signal strength, package frequency, package loss vs reliability from both pole stations and source will determine the power consumption of the base station and the system. The base station will never be in idle state and it must be able to reach pole station resulting in the highest cost function of the system.


\subsection{Sink: Base Station}\label{sc:sinkBaseStation}
The required detail of information needed to give a good user estimate will raise the question of acceptable package loss. Signal strength, package frequency, package loss vs reliability from both pole stations and source will determine the power consumption of the base station and the system. The base station will never be in idle state and it must be able to reach pole station resulting in the highest cost function of the system.
