\section{Energy consumption}\label{sc:powerConsumption}
\subsection{Runner node}
%TODO: Review this. What does it mean "Different levels of energy consumption will be determined by the chosen protocol"? WIll protocol work based on whats best energy-wise or tell us how much energy we use?
The base station will be requesting data from the runner at a periodic transmission rate. Different levels of energy consumption will be determined by the chosen protocol, but in any case, the power consumption will be considered, over a period of one second, constant. The less energy consumption of the runner will give longer individual lifetime and runtime for it, but low signal strength of node A might not give a lowest possible system power consumption. Depending on the needed quality of the received package, e.g. -84dBm, a cut of distance will be calculated and measured. Distance measuring will be limited by interference providing a need for a scalable transfer function estimate and an average over multiple measurements. Life time of node A will be considered when half of the battery capacity is used.

\subsection{Base station, north and south relay station}

Idle time, receiving and transmitting power consumption will be calculated and measure. When out of range the “pole” stations will in theory go to an idle state to save power. When in transmitting mode different measurements will be conducted depending on the chosen protocol. E.g. firm or no handshakes between pole station and base station will be measured leading to different possible distances between jumps.