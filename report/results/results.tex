\chapter{Results}\label{ch:results}

Using the test setup and mechanics described in \ref{ch:testAndPerformance}, we conducted a experiment of four different scenarios. We wanted to see whether the hopping frequency would increase, when we increased the circumference of the track, evaluate the packet sent/loss ratio and measure the overall perception of data. The test results would indicate if our hopping algorithm, the implemented stop-and-wait ARQ protocol and distance measurements were correct and working. All results are gathered from the base station.

\noindent To quickly examine the parameters measured in the test runs:

\noindent \textbf{\textit{n} request packets sent ($p_1$):} Number of packets sent from the base station over the course of the test. These include retries due to ARQ timers expiring. View this number as "the times we have asked for data".

\noindent \textbf{\textit{n} packets relayed to node 1 ($p_2$):} Number of packets relayed to north relay station.

\noindent \textbf{\textit{n} packets relayed to node 2 ($p_3$):} Number of packets relayed to south relay station.

\noindent \textbf{\textit{n} ACK's received ($p_4$):} Number of acknowledgments received by the ARQ protocol. The closer this number is to the packets sent, the more stable the data link connection between our endpoints is.

\noindent \textbf{\textit{n} DATA's received ($p_5$):} Number of data packets received. Ideally, this should be close to the packets sent as well. If $p_5$<$p_1$, then $p_1$-$p_5$ request packet.

\noindent \textbf{\textit{n} packets not acknowledged in time ($p_6$):} Number of packets not acknowledged before the ARQ timer ran out. This is strongly related to the timers on the base station and heavily influenced by interference and signal noise. Also in our test set-up we tried to get as much data from the runner as possible, so timers were strict.

\noindent Table \ref{table:datascenarios} shows the final result of each completed scenario lasting 48 minutes each, which equals about 144 runs with the train. Each minute we recorded parameters $p_1$ to $p_6$. All are initially set to zero. We decided not to change the RSSI threshold in each scenario.

\begin{table}[H]
	\centering
	\begin{tabular}{|l|l|l|l|l|l|l|} \hline
		Sn. & \pbox{18cm}{$p_1$} & \pbox{18cm}{$p_2$} & \pbox{18cm}{$p_3$} & \pbox{18cm}{$p_4$} & \pbox{18cm}{$p_5$} & \pbox{18cm}{$p_6$} \\ \hline
		1 & 29148 & 4066 & 3512 & 22169 & 34167 & 20700 \\ \hline
		2 & 29838 & 8115 & 8499 & 17880 & 23563 & 21111 \\ \hline
		3 & 29258 & 3706 & 4021 & 21944 & 32830 & 21244 \\ \hline
		4 & 30523 & 1232 & 8425 & 19137 & 28874 & 23188 \\ \hline
	\end{tabular}
	\label{table:datascenarios}
	\caption{Data from scenarios one-four completed with 144 runs.}
\end{table}

\noindent As expected, the results vary depending on the track length and the channel used. When using a length of 56 centimeters (scenario two and four), we see $p5$=<$p_1$, meaning the base station received the same or less data packets than requested. In scenario four they are even fairly close. If $p5$>$p_1$, it could be due to fading or missed timers. Also worth noting is the increased use of relays ($p_2$ and $p_3$) when the length is 56cm. Changing channels between 11 and 4 does not seem to increase the amount of data packets received or lower the number not acknowledged in time.

\noindent Figure \ref{fig:noackreceived} shows the correlation between the number of ACK's received, $p_4$, at the base station per packet sent from it. We would seek these values to approximate each other. Scenario 2 received a large number of ACK's around the 24408 packet sent mark, perhaps due to deep fading.

\begin{figure}[H]
	\centering
	\includegraphics[width=0.8\linewidth]{results/NoAckReceived}
	\caption[No. ACK's received per packet sent.]{}
	\label{fig:noackreceived}
\end{figure}

\noindent Figure \ref{fig:nodatareceived} shows the number of data packets received , $p_5$, from the runner node either direct or via one of the relays per packet sent. Scenario 1 and 3 (length=43.5cm) follows each other, which means that changing channel from 11 to 4 does not have a impact on the received number of data packets when using length=43.5. However changing the length from 43.5cm to 56cm does. Scenario 4 looks to be the most successful one, as the number of data packets received is close to the number of packets (requests for data) sent.

\begin{figure}[H]
	\centering
	\includegraphics[width=0.8\linewidth]{results/NoDataReceived}
	\caption[No. DATA's received per packet sent]{}
	\label{fig:nodatareceived}
\end{figure}

\noindent Figure \ref{fig:nopacketsrelayed} shows the number of data relayed to relay station north and south combined per packet sent. As expected, scenario 2 and 4 (length=56cm) relays more packages, now that the runner node will be out of reach from the base station for a longer period of time. It might even be that neither can reach the runner due to deep fading.

\begin{figure}[H]
	\centering
	\includegraphics[width=0.8\linewidth]{results/NoPacketsRelayed}
	\caption[No. packets relayed per packet sent.]{}
	\label{fig:nopacketsrelayed}
\end{figure}

\noindent Figure \ref{fig:nopacketsnotackintime} shows the number of packets not acknowledged over the data link in time per packet sent. All scenarios follows each-other, which points to a systematically error either due to strict timers or fading issues. An error that channel or length does not seem to affect. We noticed during the live testing that more ARQ errors occurred when communicating directly rather than relaying.

\begin{figure}[H]
	\centering
	\includegraphics[width=0.8\linewidth]{results/NoPacketsNotACKInTime}
	\caption[No. packets not ACK'ed in time.]{}
	\label{fig:nopacketsnotackintime}
\end{figure}

\noindent Figure \ref{fig:nopacketsrelayedscenario2} and \ref{fig:nopacketsrelayedscenario4} show how the two scenarios that relay the most packets (2 and 4) distribute the data among our two relay station per packet relayed.

\begin{figure}[H]
	\centering
	\includegraphics[width=0.8\linewidth]{results/NoPacketsRelayedScenario2}
	\caption[No. packets relayed to north and south node in sceario 2]{}
	\label{fig:nopacketsrelayedscenario2}
\end{figure}

\begin{figure}[H]
	\centering
	\includegraphics[width=0.8\linewidth]{results/NoPacketsRelayedScenario4}
	\caption[No. packets relayed to north and south node in sceario 4]{}
	\label{fig:nopacketsrelayedscenario4}
\end{figure}

