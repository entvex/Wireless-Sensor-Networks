\chapter{Discussion}

There are a few things that could have been done to improve various parameters in this project. Some could be entire projects in their own. To improve battery life of the relays stations and the runner, one could implement a medium access (MAC) protocol such as the S-MAC that make nodes periodically sleep and auto-synchronize their sleep schedules. As we have shown, solely using the antenna uses a lot of energy and the S-MAC would help mitigate challenge in a WSN. Because of the way our case works, we have an estimated position of the runner at time $t$, which we can exploit to our advantage. When the runner is at the north area of the track therefore being covered by the north relay station, it makes little sense to have the south relay turned on and consuming energy. A way to optimize this would be to send a message to the relay not in use and make it turn off its antenna and enter sleep mode, then wake it up later when the runner is within range of that station.

%TODO: Why exactly would that make the network more scaleable? Explain :)
\noindent If we made the relay nodes able to relay between each other then that would open of the network to be scale of network. This would make it able to cover a even bigger track. Then we could also make some nodes to enter sleep mode because we would know the runner was on the other side of the track.

\noindent Our protocol in the current state is susceptible to attackers who takes advantage of the fact that our base station will keep using a relay if it can continue to give a good RSSI and heart rate data. This abuse would stop our wireless sensor network from operating correctly.