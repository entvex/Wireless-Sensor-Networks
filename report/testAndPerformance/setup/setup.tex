\section{Setup}\label{sc:setup}

To test our original scenario described in the introduction chapter \ref{ch:introduction} on \pageref{ch:introduction}, we built a smaller version of the running track as a toy train track. To get a constant speed of the runner we used a Brio\texttrademark train to symbolize a human runner that runs a track. The speed of this individual was set to $12\frac{km}{hr}$, but the Brio train had a speed of $0.2604\frac{km}{hr}$.

\noindent The track we used in the theory chapter \ref{ch:theory} was 906.17m long and the runner would run a marathon which is 46.56km and therefore equals 46.56 rounds. Using the standard Brio track pack we build a track with length of 43.5cm. In the theory we send $4\frac{packets}{sec}$ and that means our time per packet should be $\frac{1}{4sec} = 0.250ms$, thus about $1.087*10^3$ packets per round. But as our train is vastly slower than an actual runner, and the train track was shorter than the running track, we needed to account for the differences between our test setup and the theory we had calculated. Our initial calculations showed that we needed to send XXX packets and that was not possible due to limits of the hardware, as seen in Figure POWERFIFG REF it takes 15ms to send a packet without accounting for other tasks going on at the telosb, eg. handling computations. We divided the packets per round by 4 $ \frac{packetsPerRound}{4} = 271.852 $ as a result needing $57.016 ms$ per packet for the hardware to handle the speed. A picture of the setup can be seen in figure \ref{fig:testSetup}. For detailed calculations, please see appendix XXX.
%TODO: Fix XXX packets
%TODO: POWERFIFG REF