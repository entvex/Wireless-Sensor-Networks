%%%%%%%%%%%%%%%%%%%%%%%%%%%%%%%%%%%%%%%%%
% Journal Article
% LaTeX Template
% Version 1.4 (15/5/16)
%
% This template has been downloaded from:
% http://www.LaTeXTemplates.com
%
% Original author:
% Frits Wenneker (http://www.howtotex.com) with extensive modifications by
% Vel (vel@LaTeXTemplates.com)
%
% License:
% CC BY-NC-SA 3.0 (http://creativecommons.org/licenses/by-nc-sa/3.0/)
%
%%%%%%%%%%%%%%%%%%%%%%%%%%%%%%%%%%%%%%%%%

%----------------------------------------------------------------------------------------
%	PACKAGES AND OTHER DOCUMENT CONFIGURATIONS
%----------------------------------------------------------------------------------------

\documentclass[twoside,twocolumn]{article}

\usepackage{blindtext} % Package to generate dummy text throughout this template 

\usepackage[sc]{mathpazo} % Use the Palatino font
\usepackage[T1]{fontenc} % Use 8-bit encoding that has 256 glyphs
\linespread{1.05} % Line spacing - Palatino needs more space between lines
\usepackage{microtype} % Slightly tweak font spacing for aesthetics

\usepackage[english]{babel} % Language hyphenation and typographical rules

\usepackage{graphicx}
\usepackage{amsmath}
\usepackage{txfonts}
\usepackage{tabularx}
\usepackage{listings}
\usepackage{float}
\usepackage{pbox}
\usepackage{color}
\usepackage{todonotes}

\usepackage[hmarginratio=1:1,top=32mm,columnsep=20pt]{geometry} % Document margins
\usepackage[hang, small,labelfont=bf,up,textfont=it,up]{caption} % Custom captions under/above floats in tables or figures
\usepackage{booktabs} % Horizontal rules in tables

\usepackage{lettrine} % The lettrine is the first enlarged letter at the beginning of the text

\usepackage{enumitem} % Customized lists
\setlist[itemize]{noitemsep} % Make itemize lists more compact

\usepackage{abstract} % Allows abstract customization
\renewcommand{\abstractnamefont}{\normalfont\bfseries} % Set the "Abstract" text to bold
\renewcommand{\abstracttextfont}{\normalfont\small\itshape} % Set the abstract itself to small italic text

\usepackage{titlesec} % Allows customization of titles
\renewcommand\thesection{\Roman{section}} % Roman numerals for the sections
\renewcommand\thesubsection{\roman{subsection}} % roman numerals for subsections
\titleformat{\section}[block]{\large\scshape\centering}{\thesection.}{1em}{} % Change the look of the section titles
\titleformat{\subsection}[block]{\large}{\thesubsection.}{1em}{} % Change the look of the section titles

\usepackage{fancyhdr} % Headers and footers
\pagestyle{fancy} % All pages have headers and footers
\fancyhead{} % Blank out the default header
\fancyfoot{} % Blank out the default footer
\fancyhead[C]{To Hop or Not to Hop $\bullet$ May 2018 $\bullet$ Vol. I, No. 1} % Custom header text
\fancyfoot[RO,LE]{\thepage} % Custom footer text

\usepackage{titling} % Customizing the title section
\usepackage{hyperref} % For hyperlinks in the PDF

\lstset{ %
	language=C,                % choose the language of the code
	basicstyle=\footnotesize,       % the size of the fonts that are used for the code
	numbers=none,                   % where to put the line-numbers
	numberstyle=\footnotesize,      % the size of the fonts that are used for the line-numbers
	stepnumber=1,                   % the step between two line-numbers. If it is 1 each line will be numbered
	backgroundcolor=\color{white},  % choose the background color. You must add \usepackage{color}
	showspaces=false,               % show spaces adding particular underscores
	showstringspaces=false,         % underline spaces within strings
	showtabs=false,                 % show tabs within strings adding particular underscores
	frame=single,           % adds a frame around the code
	tabsize=2,          % sets default tabsize to 2 spaces
	captionpos=b,           % sets the caption-position to bottom
	breaklines=true,        % sets automatic line breaking
	breakatwhitespace=false,    % sets if automatic breaks should only happen at whitespace
	escapeinside={\%*}{*)}          % if you want to add a comment within your code
}

%----------------------------------------------------------------------------------------
%	TITLE SECTION
%----------------------------------------------------------------------------------------

\setlength{\droptitle}{-4\baselineskip}

\pretitle{\begin{center}\Huge\bfseries}
\posttitle{\end{center}}
\title{Wireless Sensor Networks \\ To Hop or Not to Hop} % Article title
\author{
\textsc{David Jensen} \\[1ex]
\normalsize Stud. M.Sc. CE, ASE, Aarhus University \\
\href{mailto:11229@post.au.dk}{11229@post.au.dk}
\and
\textsc{Henrik Bagger Jensen} \\[1ex]
\normalsize Stud. M.Sc. CE, ASE, Aarhus University \\
\normalsize \href{mailto:201304157@post.au.dk}{201304157@post.au.dk}
\and
\textsc{Christian M. Lillelund} \\[1ex]
\normalsize Stud. M.Sc. CE, ASE, Aarhus University \\
\normalsize \href{mailto:201408354@post.au.dk}{201408354@post.au.dk}
\and
\textsc{Troels Thomsen} \\[1ex]
\normalsize Stud. M.Sc. ET, ASE, Aarhus University \\
\normalsize \href{mailto:09641@post.au.dk}{09641@post.au.dk}
}
\date{\today} % Leave empty to omit a date
\renewcommand{\maketitlehookd}{%
\begin{abstract}
\noindent Direct or relayed communication between nodes in a wireless sensor network is one of the corner stones of electrical and computer engineering. The life time and reliability of wireless nodes often becomes the bottle neck of pioneering ideas, so the usage of multiple nodes to sustain efficiency can be a necessity. Typically, when a consumer node cannot reach a wireless router in a wireless local area network (WLAN), other fixed access points must relay the packets of the consumer. In a wireless sensor network (WSN), other concerns also come into play, such as network dynamicity, battery levels and noise interference, which must be considered before relaying packets. In this project a relaying protocol was build based on signal strength and accumulated transmission errors (lost packets) The protocol tries to predict when to change between direct communication and relaying, while the project tries to validate the quality of the designed and implemented protocol for the different conditions of transmission range and channels. The protocol quality is determined in terms of the event quality (how many times did we notice an event), energy consumption aiming to use the least amount of energy to extend lifetime, or conceivably a compromise of both. Events in our case is the heart rate of an athlete runner running an oval-formed running track. We find that creating such a multivariate decision protocol is indeed possible and results show that the effect on event quality overshadow the drawback of intensified energy consumption when increasing the length of the running track.

%Whether to relay data packets or to send them directly between two computers in a wireless network setting is a frequently discussed topic in computer engineering. Typically, when a consumer node cannot reach a wireless router in a wireless local area network (WLAN), other fixed access points in place must relay the packets of the consumer. In a wireless sensor network (WSN), other concerns come into play, such as network dynamicity, battery levels and noise interfence, which must be considered before relaying packets. In this project we build a relaying protocol based on signal strength and accumulated transmission errors (lost packets) and use it to investigate what factors it pays for a WSN to relay by: The event quality (how many times did we notice an event), energy consumption aiming to use the least amount of energy to extend lifetime, or conceivably a comprimise of both. Events in our case is the heart rate of an athlete runner running a oval-formed running track. We find that creating such a multivariate decision protocol is indeed possible and our results show XXX but using signal strentgh values from the CC2420 radio can be unreliable and that a wireless sensor node should only relay packets when it is absolutely necessary to reach a destination in order to save energy.
\end{abstract}
}

%----------------------------------------------------------------------------------------

\begin{document}

% Print the title
\maketitle

%----------------------------------------------------------------------------------------
%	ARTICLE CONTENTS
%----------------------------------------------------------------------------------------

\chapter{Introduction}

Scenario introduction

Protocol introduction, To hop or not to hop, what are we going to check (power consumption and signal strength)

Power consumption test, battery lifetime, overall system lifetime compared to single node lifetime etc.

Signal strength test, when will deep fading occur, when should we jump compared to package loss etc.

\chapter{Theory}\label{ch:theory}

This chapter will introduce theory concepts used in this project. We will cover both antenna theory, go into detail about fading and explain how the ARQ method works.

\section{Protocol}\label{th:theory}
bla bla Protocol

%Sorry this is prop wrong LaTeX. But the figoure won't work.
\includegraphics[width=0.7\linewidth]{theory/protocol/toHopOrNotArqSequence}\label{fig:tohopornotarqsequence}

\chapter{Implementaition}\label{ch:implementation}

% Introduction to implementation

bla bla please note that all the codesnippets for the Implementaition have been turned into pseudo code to ease readability. For the actual nesC code please see the appendix  

\section{Base station}\label{sc:basestation}

The base station is master node in our setup. It is responsible for three import tasks: Initiating data requests, decide if the runner is out of range and keep track of past events. Figure \ref{fig:tohopornotarqsequence} shows the overall flow of data, but we shall examine the more detailed parts here.

\noindent The main control loop is started by a timer every $s$ second. It asserts if the runner is deemed out of range by calling a function and uses the feedback (either true or false) to increase an error counter that is used to change destination node. In other words, when a certain number of errors have occurred on the current link, we change request destination (from directly to relaying or vice versa). Listing \ref{lst:basestation1} is an example.
\noindent
\begin{minipage}[t]{0.95\linewidth}
	\begin{lstlisting}[language=Python, numbers=none, caption=XXX, label={lst:basestation1}]
Timer0.fired() {
	if out of range and link is direct {
		if the error count is below max
			use_next_destination
		else
			increase_error_count
	}
	if link is direct
		send_a_message_direct
	if link is relay to node north
		send_a_message_to_north
	if link is relay to node south
		send_a-message_to_south
}
	\end{lstlisting}
\end{minipage}

\noindent Next is how the base station determines if the runner is out range. When new replies are received directly from the runner (it starts in range) we save the RSSI\footnote{RSSI is a scalar register value on the CC2420 radio calculated from the RF input power in dBm.} value in a FIFO queue with a length of 10. In such a queue, new data is inserted at back and taken out from the front. This means that the latest RSSI value of the runner is the back entry, with $n=0,1,2 ... 8$ being previous positions. Our algorithm takes a mean of these and multiplies it with a weighted score. If the latest position is lower than the mean, it is added to the weighted mean of the previous positions. If it is greater the average is then subtracted from it. The result constitutes a new estimated position of the runner. Listing \ref{lst:basestation2} is an example.

\noindent
\begin{minipage}[t]{0.95\linewidth}
\begin{lstlisting}[language=Python, numbers=none, caption=XXX, label={lst:basestation2}]
bool isOutOfRange {
	if current size of queue is not max
		return
	for(i = 0; i < queue_size; i++)
		previousPos += queue_part[i]
		
	lastPos = queue_back_entry
	mean = previousPos/queue_size
	
	if lastPos is less than mean
		newPos = (lastPos*1)+(mean*0.1);
	else
		newPos = (lastPos*1)-(mean*0.1);
	
	if(newPos is larger than threshold)
		return true;
	return false;
}
\end{lstlisting}
\end{minipage}

\noindent The weights used can be changed to put more empathize on the previous positions or more on the latest.


\includegraphics[width=0.7\linewidth]{theory/protocol/toHopOrNotArqSequence}
\label{fig:tohopornotarqsequence}

\subsection{Relay}\label{sc:relay}
The relay stations in this setup are slaves to the base station. Their functionality is to relay messages to the runner node, where the base station node is out of reach. Because of this the relay station listens to the network and in case of being spoken to, it replies with an acknowledge (ACK) and then sends a request for data to the runner node. In case it is not able to receive contact from the runner after three times, it sends an error message to the base station.

\begin{minipage}[t]{0.95\linewidth}
	\begin{lstlisting}[caption=Receive message event of Relay., label={lst:relay1}]
event message_t * Receive.receive(message_t *msg, void *payload, uint8_t len){
	if (len == sizeof(requestMessage)) {
		requestMessage* reqmsg = (requestMessage*) payload;
		
		if (reqmsg->relayNodeid == TOS_NODE_ID) {
			// Send Acknowledge
			sendAcknowledge(reqmsg);
			
			if (reqmsg->data == 0) {	
				requestFromBase = *reqmsg;
				call Timer0.startPeriodic(TIMER0_PERIOD_MILLI);		
			}
			else {
				requestFromRunner = *reqmsg;
				call Timer1.startPeriodic(TIMER1_PERIOD_MILLI);
			}
		}
	}
	\end{lstlisting}
\end{minipage}

\noindent On listing \ref{lst:relay1} the primary functionality of the two relay stations can be seen. The relay stations are always in need of commands to them before sending a command themselves, they will not work on their own and therefore relies on requests from the base station and answers from the runner.

\section{Runner}\label{sc:runner}
The runner nodes task in our wireless sensor network is to respond to requestMessages that can be sent from basetation or the relay nodes. As seen in the seqvens digram shown in figure \ref{fig:tohopornotarqsequence}

\section{Energy Lab}\label{sc:energylab}

To test the 
\chapter{Test and Performance}\label{ch:testAndPerformance}

\section{Setup}\label{sc:setup}
To test our original scenario described in the Introduction on page \pageref{ch:introduction} we built a smaller version of the running track. To get a constant speed and a track to run tests on we use a Brio\texttrademark train to symbolism a human runner that runs a track. The speed of the runner was set to $12\dfrac{km}{hr}$ but the Brio train had a speed of $0.2604\dfrac{km}{hr}$. The track we used in the theory part was 906.17m long and the runner would run a marathon which is 46.56km and therefor is 46.56 rounds. Using the standard Brio track pack we build a track with length of 43.5cm. In the theory we send $4\dfrac{packets}{sec}$ and that means our time per packet should be $\dfrac{1}{4sec} = 0.250ms$ and therefor about $ 1.087*10^3$ packets per round. 

 so we needed to account for the differences between our test setup and the theory made in section \ref{table:datascenarios} on page \pageref{table:scenarios}. So 





 and 56cm. 


\begin{figure}[H]
	\centering
	\includegraphics[width=1\linewidth]{testAndPerformance/setup/setup}
	\caption{The train track withbasestation, north and south relay nodes and the runner. }
	\label{fig:testSetup}
\end{figure}



\begin{table}[H]
	\centering
	\begin{tabular}{|l|l|l|l|l|l|l|} \hline
		Scenario & \pbox{18cm}{Channel} & \pbox{18cm}{RSSI} & \pbox{18cm}{Length of track} \\ \hline
		1 & 11 & -40 & 43.5cm \\ \hline
		2 & 11 & -40 & 56cm \\ \hline
		3 & 4 & -40 & 43.5cm \\ \hline
		4 & 4 & -40 & 56cm \\ \hline
	\end{tabular}
	\label{table:scenarios}
	\caption{Four test scenarios using channel 11/4 and different track length.}
\end{table}


					
\chapter{Results}\label{ch:results}

Using the test setup and mechanics described in \ref{ch:testAndPerformance}, we conducted a experiment of four different scenarios. We wanted to see whether the hopping frequency would increase, when we increased the circumference of the track, evaluate the package sent/loss ratio and measure the overall perception of data. The test results would indicate if our hopping algorithm, the implemented ARQ protocol and distance measurements were correct and working. The four scenarios are described in chapter \ref{ch:testAndPerformance}.											
\section{Discussion}

There are a few things that could have been done to improve various parameters in this project. Some could be entire projects in their own. To improve battery life of the relays stations and the runner, one could implement a medium access (MAC) protocol such as the S-MAC that make nodes periodically sleep and auto-synchronize their sleep schedules. As we have shown, solely using the antenna uses a lot of energy and the S-MAC would help mitigate challenge in a WSN. Because of the way our case works, we have an estimated position of the runner at time $t$, which we can exploit to our advantage. When the runner is at the north area of the track therefore being covered by the north relay station, it makes little sense to have the south relay turned on and consuming energy. A way to optimize this would be to send a message to the relay not in use and make it turn off its antenna and enter sleep mode, then wake it up later when the runner is within range of that station.

\noindent If we made the relay station able to relay between each-other then that would increase the scalability of the network. The stations would be able to cover a even bigger track if stations were relaying messages by a greedy algorithm pattern always using the shortest path to the runner. With this we could also make some nodes enter sleep mode because we would know when the runner was on the other side of the track and relay on other stations to relay messages to that part.

\noindent Our protocol in the current state is susceptible to attackers who would take advantage of the fact that our base station will keep using a relay if it can continue to provide heart rate data. The attacker could abuse this by performing a man-in-the-middle attack acting as a legitimate relay station and either just eavesdropping, drop the packets or falsify contained information. This could stop our wireless sensor network from operating correctly.								      
\section{Conclusion}


RSSI is bad when measuring distances.
RSSI is bad\footnote{\cite{Heurtefeux2012}}		


%----------------------------------------------------------------------------------------
%	REFERENCE LIST
%----------------------------------------------------------------------------------------

\bibliographystyle{plainnat}
\bibliography{bibliography/Mendeley/mendeley} \label{ch:bibliography}			

%----------------------------------------------------------------------------------------

\end{document}
