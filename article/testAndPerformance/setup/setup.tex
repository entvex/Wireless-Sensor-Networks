\section{Setup}\label{sc:setup}

To test our original scenario described in the introduction section \ref{ch:introduction} on page \pageref{ch:introduction}, we built a smaller version of the running track as a toy train track. To get a constant speed of the runner we used a Brio train to symbolize a human runner that runs a track. The speed of this individual was set to $12\frac{km}{hr}$, but the Brio train had a speed of $0.2604\frac{km}{hr}$.

\noindent The track we used in the theory chapter \ref{ch:theory} was $906.17m$ long and the runner would run a marathon which is $46.56km$ and therefore equals $46.56$ rounds. Using the standard Brio track pack we build a track with length of $1.12m$. In the theory we send $4\frac{packets}{sec}$ and that means our time per packet should be $\frac{1}{4sec} = 0.250ms$, thus about $1087$ packets per round if the runner runs at an average speed of $12km/h$. But as our train is vastly slower than an actual runner, and the testing train track was shorter than the running track, we needed to account for the differences between our test setup and the theory we had calculated. Our initial calculations showed that we needed to send packets with a frequency of $14ms$ and that was not possible due to limits of the hardware. As seen in figure \ref{fig:radioOn_sendLowSignal} it takes $10ms$ to send a packet without accounting for other tasks going on in the telosb node, e.g. handling computations. We divided the packets per round by 4; $\frac{1087}{4} = 272$. As a result needing to send a packet every $(\frac{272}{15.5})^{-1}$ = $57.016 ms$. It takes the train $15.5$ seconds to run one lap. A picture of the setup can be seen in figure \ref{fig:testSetup}. For detailed calculations, please see appendix XXX.
%TODO: Fix XXX packets