\section{Conclusion}\label{ch:conclusion}

The focus of this project has been to build a WSN protocol that would relay packets based on accumulated transmissions errors and the received signal strength from RF input power measurements. The protocol was designed for a real-life running track with a base station actively requesting heart rate data from a runner carrying a sensor node, however for testing purposes we scaled parameters including track length, distance between nodes and the request interval to match a practical arrangement of a toy train driving a short oval-shaped track. 

\noindent Our results show, that the protocol is able to relay packets when required to reach the runner node, but not all packets gets acknowledged in time by the data-link ARQ protocol built on top. This is possibly due to strict timings on the sender's part. We show with conducted measurements that the antenna (CC2420) can be very energy-consuming, while processor computations on the telosb are far cheaper. Therefore we conclude, that packet hopping should be kept at a minimum whenever possible to avoid additional energy dissipation.

\noindent Regarding the use of RSSI to relay packets, we echo the thoughts of \cite{Heurtefeux2012} and find that the readings from CC2420 can be fairly inaccurate and do not represent a clear picture when predicting whether the next packet in a data stream will be lost, thus additional features should be included in the decision-making.