\section{Discussion}
\noindent In the test scenario we see a lot more packets sent than what we in theory expected, as calculated in section \ref{ch:results} Results. It is hard to confirm the exact problem, however because the nodes are sending at speeds of $20ms$, it is easily assumed that there are problems with the hardware not having time to send acknowledges before a new request is send. This is the results from trying to emulate the report scenario in a smaller scenario with the train track, where the train speed does not comply to the other factors of the field, resulting in too high a requirement of the node hardware, to be able to give theoretical results. Changing to a correct scenario of a runner on a running track, will give very different results.

\noindent Under the test, it is shown that there is a higher percentage number of relayed messages than what was assumed in theory, as calculated in section \ref{ch:results} Results. The reason behind, is that the protocol was implemented with the knowledge of understanding when to change from direct to relay messaging effectively, but not the other way around. In short the base station knows when to change from directly talking to the runner node to relaying the message through a relay node, but does not get information on when to change back again before the relay is out of reach of the runner node.

\noindent There are a few things that could have been done to improve various parameters in this project. Some could be entire projects in their own. To improve battery life of the relays stations and the runner, one could implement a medium access (MAC) protocol such as the S-MAC that make nodes periodically sleep and auto-synchronize their sleep schedules. As we have shown, solely using the antenna uses a lot of energy and the S-MAC would help mitigate challenge in a WSN. Because of the way our case works, we have an estimated position of the runner at time $t$, which we can exploit to our advantage. When the runner is at the north area of the track therefore being covered by the north relay station, it makes little sense to have the south relay turned on and consuming energy. A way to optimize this would be to send a message to the relay not in use and make it turn off its antenna and enter sleep mode, then wake it up later when the runner is within range of that station.

\noindent If we made the relay stations able to relay between each-other then that would increase the scalability of the network. The stations would be able to cover a even bigger track if they were relaying messages by a greedy algorithm pattern always using the shortest path to the runner.

\noindent Our protocol in the current state is susceptible to attackers who would take advantage of the fact that our base station will keep using a relay if it can continue to provide heart rate data. The attacker could abuse this by performing a man-in-the-middle attack acting as a legitimate relay station and either just eavesdropping, drop the packets or falsify contained information.

\noindent When looking at the energy consumption, we can conclude that the best method of saving energy is to shut down the radio when not needed, if it can be off in enough time that the overshoot, when the radio is turned on, will not be more expensive than having the radio on at all time. There are also ways of acquiring energy rather than just saving it. In this scenario using solar energy and passive human power generation methods are viable, since the nodes are outside at day and the runner node is attached to a human.\footnote{\cite{Sudevalayam2011}}
