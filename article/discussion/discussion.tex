\section{Discussion}

There are a few things that could have been done to improve various parameters in this project. Some could be entire projects in their own. To improve battery life of the relays stations and the runner, one could implement a medium access (MAC) protocol such as the S-MAC that make nodes periodically sleep and auto-synchronize their sleep schedules. As we have shown, solely using the antenna uses a lot of energy and the S-MAC would help mitigate challenge in a WSN. Because of the way our case works, we have an estimated position of the runner at time $t$, which we can exploit to our advantage. When the runner is at the north area of the track therefore being covered by the north relay station, it makes little sense to have the south relay turned on and consuming energy. A way to optimize this would be to send a message to the relay not in use and make it turn off its antenna and enter sleep mode, then wake it up later when the runner is within range of that station.

\noindent If we made the relay station able to relay between each-other then that would increase the scalability of the network. The stations would be able to cover a even bigger track if stations were relaying messages by a greedy algorithm pattern always using the shortest path to the runner. With this we could also make some nodes enter sleep mode because we would know when the runner was on the other side of the track and relay on other stations to relay messages to that part.

\noindent Our protocol in the current state is susceptible to attackers who would take advantage of the fact that our base station will keep using a relay if it can continue to provide heart rate data. The attacker could abuse this by performing a man-in-the-middle attack acting as a legitimate relay station and either just eavesdropping, drop the packets or falsify contained information. This could stop our wireless sensor network from operating correctly.

\noindent When looking at the energy consumption, we can conclude that the best method of saving energy is to shut down the radio when not needed, if it can be shut down in enough time that the overshoot, when the radio is turned on, won't be more expensive than having the radio on at all time. But there are also ways of acquiring energy rather than just save it, especially using solar energy or passive human power generation methods are a very viable methods when working with WSNs, when the nodes are outside at day and where humans are involved\footnote{\cite{Sudevalayam2011}}.
